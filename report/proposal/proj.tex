\documentclass[letterpaper,twocolumn,10pt]{article}
\usepackage[bookmarks,colorlinks=true]{hyperref}

\begin{document}

%%%%%%%%%%%%%%%%%%%%%%%%%%%%%%%%%%%%%%%%%%%%%%%%%%%%%%%%%%%%%%%%%%%%%%%%%%%%%%%%
%
% Metadata
%

	\title{\Large \bf
		Towards Ubiquitous OpenPGP Hardware
	}

	\date{October 13, 2013}

	\author{
		Matt Monaco \\
		\texttt{matthew.monaco@colorado.edu}
		\and
		Andy Sayler \\
		\texttt{andrew.sayler@colorado.edu}
	}

	\maketitle

\begin{abstract}

Public key cryptography and the OpenPGP protocol are technologies which offer
advantages in both security and convenience over symmetric encryption schemes,
traditional identification schemes, and certification systems. Hardware support
for OpenPGP in the form of smartcard and USB devices allows it to be used by the
layperson in fairly wide range of settings from single sign on (SSO) to email
security to physical access in enterprise environments. However, OpenPGP
hardware is still a niche market, mainly in large enterprises.

We believe the power and flexibility of OpenPGP should be ubiquitous, replacing
a multitude of disparate systems and protocols used by individuals, small to
enterprise businesses, and governments. Furthermore, we believe the main
hindrance to OpenPGP's widespread adoption is the lack of fully featured and
accessible hardware. In this paper, we explore the current hardware offerings
and their limitations. Additionally, we look forward to the role public key
cryptography and OpenPGP might play given widespread adoption.

\end{abstract}

%%%%%%%%%%%%%%%%%%%%%%%%%%%%%%%%%%%%%%%%%%%%%%%%%%%%%%%%%%%%%%%%%%%%%%%%%%%%%%%%
%
% Main
%


%%%%%%%%%%%%%%%%%%%%%%%%%%%%%%%%%%%%%%%%%%%%%%%%%%%%%%%%%%%%%%%%%%%%%%%%%%%%%%%%
%
% Refs, etc
%

	\nocite{*}
	\footnotesize
	\bibliographystyle{abbrv}
	\bibliography{refs}

\end{document}

% vim: set noet : %
