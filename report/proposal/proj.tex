\documentclass[letterpaper,twocolumn,10pt]{article}
\usepackage[bookmarks,colorlinks=true]{hyperref}

\usepackage[text={6.5in, 9in}, centering]{geometry}

\newcommand{\mlist}[1]{\begin{itemize}#1\end{itemize}}

\begin{document}

%%%%%%%%%%%%%%%%%%%%%%%%%%%%%%%%%%%%%%%%%%%%%%%%%%%%%%%%%%%%%%%%%%%%%%%%%%%%%%%%
%
% Metadata
%

\title{\Large \bf
  Towards Ubiquitous OpenPGP Hardware
}

\date{October 13, 2013}

\author{
  Matt Monaco \\
  \texttt{matthew.monaco@colorado.edu}
  \and
  Andy Sayler \\
  \texttt{andrew.sayler@colorado.edu}
}

\maketitle

\begin{abstract}

Public key cryptography and the OpenPGP protocol~\cite{rfc4880} are
technologies which offer advantages in both security and convenience
over symmetric encryption schemes, traditional identification schemes,
and certification systems. Hardware support for OpenPGP in the form of
smartcard and USB devices allows it to be used by the layperson in
fairly wide range of settings from single sign on (SSO) to email
security to physical access in enterprise environments. However,
OpenPGP hardware is still a niche market, mainly in large
enterprises. OpenPGP capable hardware (e.g. SmartCards) can both
simplify and further secure the use of OpenPGP.

We believe the power and flexibility of OpenPGP should be ubiquitous, replacing
a multitude of disparate systems and protocols used by individuals, small to
enterprise businesses, and governments. Furthermore, we believe the main
hindrance to OpenPGP's widespread adoption is the lack of fully featured and
accessible hardware. In this paper, we explore the current hardware offerings
and their limitations. Additionally, we look forward to the role public key
cryptography and OpenPGP might play given widespread adoption.

\end{abstract}

%%%%%%%%%%%%%%%%%%%%%%%%%%%%%%%%%%%%%%%%%%%%%%%%%%%%%%%%%%%%%%%%%%%%%%%%%%%%%%%%
%
% Main
%

\section{Proposal}

For our project we would like to: provide a brief overview of the
functionality of OpenPGP; take a survey of current hardware, firmware,
and software which supports OpenPGP; investigate extensions to the
OpenPGP protocol which might be required to make it universally
useful; discuss the security and threat models of OpenPGP hardware,
and pinpoint major use-cases for systems which public key cryptography
can replace, and likely even improve.

Our initial investigation into OpenPGP hardware has turned up
offerings which all have at least one major limitation. Typically this
is in the amount of key material that they are able to store. However,
other limitations include the number of keys (regardless of size) that
can be stored, the interface, cost, and advanced features such as key
generation, temper-resistance, access control, companion RFID
chipsets, etc.

Hardware which we would consider perfect would have the following properties:

\mlist{
\item Support at least one 4096 bit RSA master key
\item Support at least three 4096 bit sub-keys
\item Support multiple encryption algorithms, but at least RSA
\item Have a completely open specification and interface
\item Have a completely open implementation
\item Support key generation
\item Be tamper resistant; self-destruct in the face of
  forensic analysis
\item Provide physical access control (e.g. an external keypad)
\item Have a USB interface, possibly exposing itself as a
  smartcard \textit{reader} with attached smartcard
\item Have a wireless/NFC interfaces for use with external
  devices (e.g. phones, door locks, etc) }

Toward these goals, we will explore the open source YubiKey
NEO~\cite{yubikey}, offered by Yubico, a company which offers small
USB-based security products. We will try to modify the Neo to meet as
many of the above requirements as possible.

We will also propose the specifications for an open source device that
goes beyond the Neo and corrects for any limitations the Neo may be
found to have.

%%%%%%%%%%%%%%%%%%%%%%%%%%%%%%%%%%%%%%%%%%%%%%%%%%%%%%%%%%%%%%%%%%%%%%%%%%%%%%%%
%
% Refs, etc
%

\footnotesize
\bibliographystyle{abbrv}
\bibliography{refs}

\end{document}

% vim: set noet : %
