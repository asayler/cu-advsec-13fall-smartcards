% SmartCard Project
% Adv. Security
% Fall 2013
% Matthew Monaco
% Andy Sayler

\documentclass[11pt, twocolumn]{article}
\usepackage[text={6.5in, 9in}, centering]{geometry}

\usepackage{epsfig}
\usepackage{url}
\usepackage{float}
\usepackage{caption}
\usepackage{subcaption}
\usepackage{color}

\usepackage{hyperref}
\hypersetup{
    colorlinks,
    citecolor=black,
    filecolor=black,
    linkcolor=black,
    urlcolor=black
}

\newenvironment{packed_enum}{
\begin{enumerate}
  \setlength{\itemsep}{1pt}
  \setlength{\parskip}{0pt}
  \setlength{\parsep}{0pt}
}{\end{enumerate}}

\newenvironment{packed_item}{
\begin{itemize}
  \setlength{\itemsep}{1pt}
  \setlength{\parskip}{0pt}
  \setlength{\parsep}{0pt}
}{\end{itemize}}

\newenvironment{packed_desc}{
\begin{description}
  \setlength{\itemsep}{1pt}
  \setlength{\parskip}{0pt}
  \setlength{\parsep}{0pt}
}{\end{description}}

\begin{document}

\title{Using SmartCards for Public Key Cryptography}

\author{
  Matt Monaco \\ \texttt{matthew.monaco@colorado.edu} \and
  Andy Sayler \\ \texttt{andy.sayler@colorado.edu}
}

\date{\today}

\maketitle

\begin{abstract}
Public key cryptography and the OpenPGP protocol are technologies
which offer advantages in both security and convenience over symmetric
encryption systems and traditional authentication
schemes. Unfortunately, standard software-based OpenPGP stacks still
exhibit usability and security issues across a range of modern use
cases. To overcome these issues, SmartCard-based OpenPGP hardware
devices allow OpenPGP to be used by the layperson in fairly wide range
of settings from single sign on (SSO) to email security to physical
access in enterprise environments. SmartCard hardware can both
simplify and further secure the use of OpenPGP. However, OpenPGP
SmartCard hardware is still a niche market, mainly in large
enterprises.

We believe the power and flexibility of OpenPGP should be ubiquitous,
replacing a multitude of disparate systems and protocols used by
individuals, businesses, and governments. Furthermore, we believe the
main hindrance to OpenPGP's widespread adoption is the lack of fully
featured and accessible hardware. In this paper, we explore the
current hardware offerings and their limitations. We propose changes
to existing hardware systems to make them more accessible by the
general use. Additionally, we explore the role public key cryptography
and OpenPGP might play given widespread adoption.
\end{abstract}

\section{Introduction}



\section{Existing Systems}

\section{Toward an Ideal Card}

\section{Future Applications}

\section{Conclusion}

\bibliographystyle{plain}
\nocite{*}
\bibliography{refs}

\end{document}
